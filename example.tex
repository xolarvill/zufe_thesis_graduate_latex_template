%!TEX program = xelatex

% ==================== 文档类选项 ====================
% 学位类型: master(硕士) 或 doctor(博士)
% 论文版本: blind(盲审) / defense(答辩) / final(正式)
\documentclass[master,final]{zufe-thesis}

% ==================== 自定义字体设置(可选)====================
% 若需使用本地 TTF 字体,请取消注释并指定路径
% \setCJKmainfont[Path=./]{SimSun.ttf}
% \setCJKsansfont[Path=./]{SimHei.ttf}
% \setCJKmonofont[Path=./]{KaiTi.ttf}
% \setmainfont[Path=./]{TimesNewRoman.ttf}

% ==================== 额外宏包 ====================
\usepackage{unicode-math} %在此加载你需要的宏包
\usepackage{lipsum}

% ==================== 参考文献 ====================
\addbibresource{example.bib} % 请确保 example.bib 存在或替换为你的 .bib 文件

% ==================== 论文基本信息(占位符)====================
\thesisTitle{基于量子纠缠模型的星际旅行路径优化研究}
\thesisTitleEN{An Investigation into Interstellar Travel Path Optimization Using Quantum Entanglement Frameworks}
\authorName{艾达·洛夫莱斯} % 示例姓名
\studentID{188800000001} % 示例学号
\mentorName{图灵·艾伦} % 示例导师
\majorName{应用密码学}
\deptName{计算科学院}
\submitDate{2047年12月}
\submitDateEN{December 2047}
\reviewDate{2047年12月} % 盲审日期
\secrecyLevel{公开} % 论文密级

% ==================== 正文开始 ====================
\begin{document}

% -------------------- 封面 --------------------
\makethesiscover

% -------------------- 声明页(盲审版自动省略)--------------------
\makestatement

% -------------------- 中英文扉页 --------------------
\makechinesetitlepage
\makeenglishtitlepage

% -------------------- 前置部分(罗马页码)--------------------
\frontmatter

% 中文摘要
\begin{abstract}
\lipsum[1-2]
\keywords{星际跃迁;量子纠缠;路径熵;黑洞导航}
\end{abstract}

% 英文摘要(关键词自动转小写)
\begin{abstracten}
\lipsum[1-2]
\keywordsen{interstellar warp; quantum entanglement; path entropy; black hole navigation}
\end{abstracten}

% 目录
\tableofcontents
% -------------------- 正文部分(阿拉伯页码)--------------------
\mainmatter

\chapter{绪论}

\section{研究背景与意义}
\lipsum[3]

\subsection{文献综述}
早期科幻作家如 \textcite{asimov1951} 设想……。

\section{研究框架}
\lipsum[4]

\chapter{理论模型}

\section{模型设定}
考虑代表性飞船在时空 \( t \) 的熵函数:
\begin{equation}\label{eq:utility}
S_t = \ln(E_t) + \kappa \cdot Q(W_t) - \psi \cdot G(W_t, W_{t-1})
\end{equation}
其中 \( E_t \) 为能量,\( W_t \) 为虫洞坐标,\( G(\cdot) \) 为引力损耗。
\begin{table}[htbp]
\centering
\caption{主要变量描述性统计}
\label{tab:stats}
\begin{tabular}{lcc}
\toprule
变量 & 均值 & 标准差 \\
\midrule
是否跃迁 & 0.07 & 0.26 \\
光年距离(百亿) & 3.9 & 2.8 \\
曲率指数 & 1.4 & 0.9 \\
\bottomrule
\end{tabular}
\end{table}
% -------------------- 后置部分 --------------------
\backmatter
% 参考文献
\printbibliography[heading=bibintoc, title=参考文献]
% 附录
\appendix
\chapter{模型推导细节}
附录公式编号为式~\ref{eq:appendix}。
\begin{equation}\label{eq:appendix}
\frac{\partial S}{\partial W} = 0
\end{equation}
\chapter{补充图表}
附录表格编号为表~\ref{tab:appendix}。
\begin{table}[htbp]
\centering
\caption{稳健性检验结果}
\label{tab:appendix}
\begin{tabular}{lc}
\toprule
变量 & 系数 \\
\midrule
维度 & 0.073*** \\
(0.022) & \\
\bottomrule
\end{tabular}
\end{table}
% 发表论文列表(盲审版自动省略)
\begin{publications}
\begin{enumerate}
\item 艾达·洛夫莱斯, 图灵·艾伦. 量子虫洞的熵最小化原理[J]. 未来物理学报, 2046(42): 314–159.
\item Lovelace, A., Turing, A. Entangled Warp Drives and Cosmic Inflation[J]. Journal of Impossible Physics, 2045, 88: 1–23.
\end{enumerate}
参与的科研项目:
\begin{enumerate}
\item 银河系边缘跃迁可行性评估. 泛星际科学基金会. 2044–2048.
\end{enumerate}
\end{publications}
% 致谢(盲审版自动省略)
\begin{acknowledgement}
本研究得以顺利完成,离不开导师的超时空指导、虚拟家人的全息支持以及同行星球学者的光子建议。特别感谢银河计算大学提供的反物质计算资源。文中任何维度塌缩,敬请超光速指正。
\end{acknowledgement}
% 封底
\makebackcover
\end{document}
```